\documentclass[aspectratio=169]{beamer}
%\documentclass{beamer}
\beamertemplatenavigationsymbolsempty
\usecolortheme{beaver}
\setbeamertemplate{blocks}[rounded=true, shadow=true]
\setbeamertemplate{footline}[page number]
%
\usepackage[utf8]{inputenc}
\usepackage[english,russian]{babel}
\usepackage{amssymb,amsfonts,amsmath,mathtext}
\usepackage{subfig}
\usepackage[all]{xy} % xy package for diagrams
\usepackage{array}
\usepackage{bm}
\usepackage{multicol}% many columns in slide
\usepackage{hyperref}% urls
\usepackage{hhline}%tables
\usepackage{mathtools}
\usepackage{graphicx}
\captionsetup[subfigure]{font={small},labelfont={small}}
\usepackage{adjustbox}
% Your figures are here:
\graphicspath{ {fig/} {../fig/} }
\newtheorem{theorem_rus}{Теорема}
\newtheorem{lemma_rus}{Лемма}
\setbeamertemplate{theorems}[numbered]
\setbeamertemplate{caption}[numbered]
\usepackage{ragged2e}
\usepackage[T1]{fontenc} % Кодировка шрифтов для западных языков
\usepackage[utf8]{inputenc} % Кодировка UTF-8
\usepackage[english]{babel} % Поддержка английского языка

\usepackage[style=authoryear,language=english]{biblatex}
\addbibresource{references.bib} % Файл с библиографией
\justifying
\input{newcommands}
%----------------------------------------------------------------------------------------------------------
\title[\hbox to 56mm{Краткое название}]{Robust Detection of AI-Generated Images}
\author[Д.\,Д.~Дорин]{Даниил Дмитриевич Дорин\\
\small Научный руководитель: к.ф.-м.н. А.\,В.~Грабовой \\
\small Ассистент: Д.\,Д.~Дорин}
\institute{Анализ данных ФПМИ МФТИ}
\date{2025}

%----------------------------------------------------------------------------------------------------------
\begin{document}
%----------------------------------------------------------------------------------------------------------
\begin{frame}
\thispagestyle{empty}
\maketitle
\end{frame}

%----------------------------------------------------------------------------------------------------------
\begin{frame}{Цель и постановка задачи}
\begin{block}{Цель работы}
    Построить модель классификации изображений на сгенерированные и реальные, устойчивую к методам генерации. 
\end{block}
\begin{block}{Постановка задачи}
Задана выборка $$\mathfrak{D} = \{\mathbf{x_i}, y_i \},\ i= 1, ..., N,$$ где $\mathbf{x_i} \in \mathbb{N}^{m \times n \times r}$ - изображение разрешения $m \times n$ с  $r$ каналами, $y_i \in \{ 0, 1\}$ \\

Строится отображение $\mathbf{F}: \mathbb{N}^{m \times n \times r} \rightarrow [0, 1] $ - отображение из изображения в вероятность того, что изображение сгенерированно.

Решается задача нахождения оптимального отображения \( \mathbf{F}^* \) в своём классе моделей \( \mathcal{F} \), т.е.:
\[
	\mathbf{F}^* = \arg\min_{\mathbf{F}^* \in \mathcal{F}} \operatorname{Log Loss}(F).
\]
\end{block}
\end{frame}
%----------------------------------------------------------------------------------------------------------
\begin{frame}{Ошибки на моделях, качество классификатора}
\begin{figure}
\centering
\includegraphics[width=0.76\textwidth]{figs/fake_images.png}
\end{figure}

\begin{figure}
\centering
\includegraphics[width=0.66\textwidth]{figs/PR_ROC.png}
\end{figure}
\end{frame}
%----------------------------------------------------------------------------------------------------------
\end{document} 