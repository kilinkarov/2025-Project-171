\documentclass[aspectratio=169]{beamer}
%\documentclass{beamer}
\beamertemplatenavigationsymbolsempty
\usecolortheme{beaver}
\setbeamertemplate{blocks}[rounded=true, shadow=true]
\setbeamertemplate{footline}[page number]
%
\usepackage[utf8]{inputenc}
\usepackage{bm}
\usepackage[english,russian]{babel}
\usepackage{amssymb,amsfonts,amsmath,mathtext}
\usepackage{subfig}
\usepackage[all]{xy} % xy package for diagrams
\usepackage{array}
\usepackage{bm}
\usepackage{multicol}% many columns in slide
\usepackage{hyperref}% urls
\usepackage{hhline}%tables
\usepackage{mathtools}
\usepackage{graphicx}
\captionsetup[subfigure]{font={small},labelfont={small}}
\usepackage{adjustbox}
% Your figures are here:
\graphicspath{ {fig/} {../fig/} }
\newtheorem{theorem_rus}{Теорема}
\newtheorem{lemma_rus}{Лемма}
\setbeamertemplate{theorems}[numbered]
\setbeamertemplate{caption}[numbered]
\usepackage{ragged2e}
\usepackage[T1]{fontenc} % Кодировка шрифтов для западных языков
\usepackage[utf8]{inputenc} % Кодировка UTF-8
\usepackage[english]{babel} % Поддержка английского языка

\usepackage[style=authoryear,language=english]{biblatex}
\addbibresource{references.bib} % Файл с библиографией
\justifying
\input{newcommands}
%----------------------------------------------------------------------------------------------------------
\title[\hbox to 56mm{Краткое название}]{Robust Detection of AI-Generated Images}
\author[Г.\,В.~Килинкаров]{Георгий Валерьевич Килинкаров\\
\small Научный руководитель: к.ф.-м.н. А.\,В.~Грабовой \\
\small Ассистент: Д.\,Д.~Дорин}
\institute{Анализ данных ФПМИ МФТИ}
\date{2025}

%----------------------------------------------------------------------------------------------------------
\begin{document}
%----------------------------------------------------------------------------------------------------------
\begin{frame}
\thispagestyle{empty}
\maketitle
\end{frame}

%----------------------------------------------------------------------------------------------------------
\begin{frame}{Цель и постановка задачи}
\begin{block}{Цель работы}
    Построить модель классификации изображений на машинно-сгенерированные и оригиальные, устойчивую к методам генерации. 
\end{block}
\begin{block}{Постановка задачи}
Задана выборка $$\mathfrak{D} = \{\bm{x_i}, y_i \},\ i= 1, ..., N,$$ где $\bm{x_i} \in \mathbb{N}_0^{H \times W \times C}$~--- изображение размера $H \times W \times C$, $y_i \in \{ 0, 1\}.$ \\

Необходимо построить отображение $\bm{F}: \mathbb{N}_0^{H \times W \times C} \rightarrow \{ 0, 1 \}.$

Для нахождения оптимального отображения \( \bm{F}^* \) в классе моделей \( \mathcal{F} \) используется Binary Cross-Entropy Loss (BCE):
\[
	\bm{F}^* = \arg\min_{\bm{F}^* \in \mathcal{F}} \operatorname{BCE}(F).
\]
\end{block}
\end{frame}
%----------------------------------------------------------------------------------------------------------
\begin{frame}{Ошибки на генерациях, качество классификатора}
\begin{figure}
\centering
\includegraphics[width=0.76\textwidth]{figs/fake_images.png}
\end{figure}

% \begin{figure}
% \centering
% \includegraphics[width=0.66\textwidth]{figs/PR_ROC.png}
% \end{figure}
\begin{figure}[ht]
    \centering
    \begin{minipage}{0.10\textwidth}
        \includegraphics[width=\linewidth]{figs/scheme.png}
    \end{minipage}
    \hspace{30pt}
    \begin{minipage}{0.35\textwidth}
        \includegraphics[width=\linewidth]{figs/PR_ROC.png}
    \end{minipage}
\end{figure}
\end{frame}
%----------------------------------------------------------------------------------------------------------
\end{document} 
