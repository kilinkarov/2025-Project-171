\documentclass{article}

\usepackage{arxiv}

\usepackage{fontspec}
\setmainfont{Linux Libertine O}

\usepackage{polyglossia}
\setdefaultlanguage{english}
\setotherlanguage{russian}

\usepackage{url}
\usepackage{booktabs}
\usepackage{amsfonts}
\usepackage{amssymb}
\usepackage{amsmath}
\usepackage{nicefrac}
\usepackage{microtype}
\usepackage{lipsum}
\usepackage{graphicx}

\usepackage[numbers]{natbib}
\usepackage{amsthm}
\usepackage{doi}
\usepackage{xcolor}

\usepackage{todonotes}
\newcommand{\todoask}[1]{\todo[color=blue!40]{\textbf{Ask:} #1}}
\newcommand{\todocheck}[1]{\todo[color=green!40]{\textbf{Check:} #1}}
\newcommand{\todoread}[1]{\todo[color=orange!40]{\textbf{Read:} #1}}
\newcommand{\todocomment}[1]{\todo[color=purple!40]{\textbf{Comment:} #1}}


\title{ Robust Detection of AI-Generated Images}

\author{%
    Kilinkarov Georgii \\
    Chair of Data Analysis\\
    Moscow Institute of Physics and Technology\\
    Moscow, Russia \\
    \texttt{kilinkarov.gv@phystech.edu} \\
    \And
    Daniil Dorin  \\
    Affiliation \\
    Address \\
    \texttt{email} \\
    \AND
    Andrey Grabovoy \\
    Affiliation \\
    Address \\
    \texttt{email}
}
\date{}

\hypersetup{
    pdftitle={Comparative Analysis of Data-Driven Approaches for Hydrological Forecasting},
    pdfsubject={cs.LG},
    pdfauthor={Eldar Khuzin, Novikov Ivan, Abramov Dmitrii},
    pdfkeywords={Flood discharge, Rainfall-runoff modeling, More}
}

\begin{document}
\maketitle

\begin{abstract}
В связи с улучшением качества машиносгенерированных изображений становится очень сложно отличать реальное изоюражение от сгенерированного.  Существующие на данный момент решения имеют низкую обощающую способность. В этой статье тестируем модели, несвязанные с нейронными сетями и изучаем распределение цветов на сгенерированных изоюражениях. Также используем всю существующую информацию и модели, для подбора наилучшего решения, делая валидацию на то, с какой именно моделью работаем. Помимо этого, используем методы графических редакторов, на основе искуственного интеллекта.
\end{abstract}

\keywords{AI-Generated Image}

\section{Introduction}
\label{sec:introduction}

В современном мире в связи с развитием генераторов изображений человеческому глазу стало уже слишком сложно отличать настоящие изображение и машиносгенерированное. Ещё сложнее человеку отличить реальное изображение от реального, но с использованием графического редактора.\cite{OnlineDetection} В связи с доступностью этих сервисов стали очень распространены разные виды мошенничества, использующие машиногенерацию. Таким образом задача детекции машинносгенерированных изображений стала очень важна.


На данный момент не существует общего подхода к решению этой задачи, устойчивого относильно появления новых моделей. Например, появление диффузионных моделей генерации изображений свело сущесвтующие на тот момент методы к точности около 60 процентов\cite{GenImage}. Таким образом, существующие на данный момент методы имеют низкую обощающую способность. Актуальные научные статьи на эту тему можно поделить на три типа: построение устойчивой модели с помощью добавления новых типов генерации в фазу обучения\cite{AivsAi, OnlineDetection}, решение задачи с помощью методов, не используюцих AI-методы (с помощью классических методов и рассмотрения спектра света)\cite{ZeroShot}, создание новых более мощных датасетов для данный задачи\cite{GenImage, CIFAKE}.


AI-модели обучается на всё более новых и новых датасетах, включая в себя новые способы генерации, создаются способы онлайн-обучения \cite{OnlineDetection}, что улучшает постепенно качество, но концептуально не отличается от предыдущих методов и не обеспечивает устойчивость в случае, если появится более иновационный метод генерации. До появления диффузионных моделей высокое качество показывал метод, рассматривающий спектр по Фурье \cite{ZeroShot}. Но на диффузионных моделях не показывает уже высокого качества.

Таким образом, в этой статье мы попробовали объеденить существующие методы и найти новый способ детекции машиносгенерированных изображений. Новизна заключается в объединении методов и построении модели, предполагающей сначала тип генерации, а потом проверяющей на генерацию уже непосредственно с предположением определенного типа генерации.



\section{Problem statement}
\label{sec:problem_statement}



На вход мы получаем $X = [X_1, ..., X_n]$ -  выборка, где $X_i \in N_{255}^{d \cdot d \cdot 3}$ - картинка. \\

Нужно построить отображение $F: N_{255}^{d \cdot d \cdot 3} \rightarrow \{ 0, 1 \} $ - отображение из картинки в её тип (реальная или сгенерированная)

Используется метрика точности(accuracy). А именно:
\[
	\operatorname{Accuracy}(f) = \frac{1}{N} \sum_{i=1}^N  I{( y_i = F(x_i))} ,
\]
где \( y_i \) --- истинное значение класса, а \( F(x_i) \) --- предсказанное значение.

Теперь нам нужно выбрать наилушчую по этой метрике модель\( f^* \) в своём классе моделей \( \mathcal{F} \), т.е.:
\[
	f^* = \arg\min_{F \in \mathcal{F}} \operatorname{MSE}(F).
\]
\bibliographystyle{unsrtnat}
\bibliography{references}

\end{document}
